\documentclass[aps,prl,twocolumn,groupedaddress]{revtex4-1}

\usepackage{amsmath}
\usepackage{amssymb}
\usepackage{float}


\begin{document}

\title{Covariance Matrix Resolution Calculations}
\author{Patrick Nave}
\maketitle

\section{Introduction}

The purpose of this guide is to provide a brief review of the mathematics and relevant experimental context using the covariance matrix resolution program.

\section{Derivation}

Consider a set of m functions $\{f_1, ..., f_m\}$ which each depend on a set of n random variables $\{x_1, ..., x_n\}$.  We define the Jacobian matrix of our system to be

\begin{equation}
J = 
\begin{bmatrix}
    \frac{\partial f_1}{\partial x_1} & \frac{\partial f_1}{\partial x_2} & \dots  & \frac{\partial f_1}{\partial x_n} \\
    \frac{\partial f_2}{\partial x_1} & \frac{\partial f_2}{\partial x_2} & \dots  & \frac{\partial f_2}{\partial x_n} \\
    \vdots & \vdots & \ddots & \vdots \\
    \frac{\partial f_m}{\partial x_1} & \frac{\partial f_m}{\partial x_2} & \dots  & \frac{\partial f_m}{\partial x_n}
\end{bmatrix}
\end{equation}

If the assumption is made that each of our n random variables is normally distributed, not necessarily independently of the others, then we have a well-defined, although possibly difficult to compute, covariance matrix $\Sigma_{\mathbf{x}}$ which describes the joint distribution.

Again operating under simplifying assumptions, we assume the covariance matrix $\Sigma_{\mathbf{x}}$ describes linear deviations, and thus we compute an approximate covariance matrix, $\Sigma$, for the m variables described by our m functions $f_k$ as follows:

\begin{equation}
\Sigma = J \Sigma_{\mathbf{x}} J^T
\end{equation}

which yields an m x m covariance matrix for our system.

\subsection{ARCS}

\textbf{Pertinent Experimental Variables:}
\begin{itemize}
\item $L_{sp} = $ (m) distance from sample to detector pixel

\item $L_{12} = $ (m) distance between beam monitors 1 and 2

\item $L_{ms} = $ (m) distance from moderator to sample position

\item $t_{12} = $ (s) time for a neutron to travel from beam monitor 1 to monitor 2

\item $t_{ms} = $ (s) time for neutron to travel from moderator to sample

\item $t_{sp} = $ (s) time for neutron to travel from sample to detector pixel

\item $v_{i,f} = $ (m/s) initial (resp., final) neutron velocity

\item $E_{i,f} = $ (m/s) initial (resp., final) neutron velocity

\item $Q_{x,y,z} = $ x (resp. y, resp. z) component of neutron wavevector in instrumental beam coordinates (z along beam, y vertical, x completing right-hand coordinate system)
\end{itemize}

\begin{table}[H]
\centering
\caption{ARCS Instrument Parameters}
\begin{tabular}{|c|c|}
\hline
Instrument Parameters & Values \\ \hline
$L_{sp}$ & (event dependent) \\ \hline
$L_{12}$ & 6.67 m \\ \hline
$L_{ms}$ & 13.60 m \\ \hline

\end{tabular}
\end{table}


\end{document}